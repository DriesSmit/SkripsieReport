\def\baselinestretch{1}
\chapter{Summary and conclusions}
\label{ch:Conclusions}

\graphicspath{{Conclusions/Figures_Conclusions/}}
\section{Project summary}

In this project an automatic test grading system is developed with the aim of grading student test using a special template. Firstly research was done into excising methods of grading test automatically. It was found that these software packages normally uses only image processing methods to graded bubbles on a paper. For this project additional machine learning capabilities was also built into the system. This allows for a better estimation of what the student  wanted to write down on the paper.

\section{How this final year project benefits society}
In the African society there is a great number of individuals who does not have access to quality educational opportunities. The educational systems these individuals do have are normally of pour grade with limited teaching assistance. Educators who are available are not accessible to learners to provide quality education. Automatic Mark Recognition software like the one developed in this project allows for a great number of tests to be assessed automatically in a short time span. This gives educators the power to handle bigger classes and thus provide more learners the opportunity for a better education.

\section{What the student learned}
During the execution of this project, the student learned that time management is important to complete an project in due time. Time management also allows an individual to continuously asses how he/she is doing with respect to a schedule. This not only increases performance, but also self confidence in the final product. Finally the student learned how to develop a software package under a deadline. This project also allowed the student to gain a basic knowledge a a broad range of fields including image processing, neural networks and probabilistic graphical models. 


\section{Future improvements}

To increase the speed of grading tests it should be considered to use a faster PGM library to inference the intended student number. This can be done by using Stellenbosch's emdw library. Further increases in test grading speed can be achieved by only doing image processing on the expected locations of the bubbles. This will bring some extra technical hurdles, but if solved can significantly increase the software's speed. Further the accuracy of the character recognition neural network can be increased by making use of Generative Adversarial Networks(GAN) to train the network on actually classified test results.

\section{Conclusion: Summary and conclusions}

For a tutorial setting with around 890 tests the system takes $\pm$ 30 minutes to grade these tests automatically. This time is longer than other OCR software, but allows for less limited answers to be given by a student. An example of these answers are the system's capability to identify a student number by only referring to the characters written in the student number box.

In conclusion a test grading system was build that can automatically grade tests with a 97.1\% accuracy. The reason for this accuracy not being at 100\% is mainly due to some students crossing out answers only partially and thus confuses the system. An additional feature is implemented that transfers tests, the system uncertain about, to a user to manually grade using the software. In combination with this manually checked tests, the system achieves an overall accuracy of 99.8\%. Thus on average only 1 test in every tutorial session of around 890 tests gets graded incorrectly.