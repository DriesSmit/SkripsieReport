\def\baselinestretch{1}
\chapter{Summary and conclusions}
\label{ch:Conclusions}

\graphicspath{{Conclusions/Figures_Conclusions/}}
\section{Project summary}
In this project an automatic test grading system was developed with the aim of grading student test using a special template. Initially, research was done into excising methods of grading these tests automatically. It was found that normally only image processing methods are used to grade bubbles on a paper. For this project additional machine learning capabilities were built into the system. This allows for a better estimation of what the student  intended to write down on the paper.

\section{How this final year project benefits society}
In the African society there is a great number of individuals who do not have access to quality educational opportunities. The educational systems these individuals do have are normally not up to standard with limited teaching assistance. Educators who are available are not always accessible to learners to provide quality education. Automatic Mark Recognition software like the one developed in this project allows for a large number of tests to be assessed automatically and accurately in a short time span. This assists educators in handling bigger classes and thus provide more learners the opportunity for a better education.

\section{What the student learned}
During the execution of this project, the student learned that time management is important to complete a project of this scale. Time management also allows an individual to continuously asses how he/she is doing with respect to a schedule. This not only increases performance, but also self confidence in the final product. Finally, the student learned how to develop a software package in a professional environment. This project also allowed the student to gain a basic knowledge on a broad range of fields including image processing, neural networks and probabilistic graphical models. 


\section{Future improvements}
To increase the speed of grading tests it should be considered to use Stellenbosch's custom PGM library, implemented in C++. By continuously updating the estimated orientation of the template as more bubble contours gets classified, accuracy in finding these bubbles can be increased. Further increases in test grading speed can be achieved by only doing image processing on the expected locations of the bubbles. This will bring some extra technical hurdles, but can increase the software's speed. Further the accuracy of the character recognition neural network can be increased by making use of Generative Adversarial Networks (GAN) to train the network on actually classified test results.

\section{Summary and conclusions}
For 890 tests the system takes approximately 30 minutes to grade these tests. This time is acceptable for the department, because it allows for more flexible answers. For example, a student number can be identified by only referring to the characters written in the student number box. The system has an overall accuracy of 97.1\% with only automatic grading. An additional feature is implemented that transfers tests which the system is uncertain about, to a user to manually grade. Combined with the human operator, only 1 test in every tutorial session (average of 890 tests) gets graded incorrectly. In combination with the manually checked tests, the system thus obtains a 99.9\% accuracy on grading tests correctly, while still allowing students greater flexible in the methods of answering these tests.