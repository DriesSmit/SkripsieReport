\chapter{Validation and results}
\label{ap:results}
\graphicspath{{Appendix5/Appendix5figures/}}

This appendix described additional test results obtained from experiments done on the systems.

\section{All tutorial results}
\label{sec:tutorialResults}

\subsection{Overview}

This automatic test grader was sucesfully used to grade 11 tutorial test in 2017. In that 11 tutorial an approximately of 99.6\% of test was graded correctly, as no corrections were made by students. The ta

\begin{table}
\caption{Description of tutorial results.} \label{tbl:tutResults}
  \centering
\begin{tabular}{|p{3cm}|p{3cm}|p{5cm}|}
\hline
\textbf{Tutorial number}&\textbf{Number of tests graded incorrectly}&\textbf{Reason for results}\\
\hline
\multicolumn{3}{|l|}{Basic system is now implemented.}\\
\hline
Tutorial 1&3&Hello assdafsadfsa dfsdfsdafdsf asdfsadfasdf sadfasdfa sdfsadfs adfasdfs dfasdfsdafsa dfsdf sadfasdf asdf asdfa dfasd fsad fsad fsa dfsdf.\\
\hline
Tutorial 2&3&Hello assdafsadfsa dfsdfsdafdsf asdfsadfasdf sadfasdfa sdfsadfs adfasdfs dfasdfsdafsa dfsdf sadfasdf asdf asdfa dfasd fsad fsad fsa dfsdf.\\
\hline
Tutorial 3&3&Hello assdafsadfsa dfsdfsdafdsf asdfsadfasdf sadfasdfa sdfsadfs adfasdfs dfasdfsdafsa dfsdf sadfasdf asdf asdfa dfasd fsad fsad fsa dfsdf.\\
\hline
Tutorial 4&3&Hello assdafsadfsa dfsdfsdafdsf asdfsadfasdf sadfasdfa sdfsadfs adfasdfs dfasdfsdafsa dfsdf sadfasdf asdf asdfa dfasd fsad fsad fsa dfsdf.\\
\hline
Tutorial 5&3&Hello assdafsadfsa dfsdfsdafdsf asdfsadfasdf sadfasdfa sdfsadfs adfasdfs dfasdfsdafsa dfsdf sadfasdf asdf asdfa dfasd fsad fsad fsa dfsdf.\\
\hline
Tutorial 6&3&Hello assdafsadfsa dfsdfsdafdsf asdfsadfasdf sadfasdfa sdfsadfs adfasdfs dfasdfsdafsa dfsdf sadfasdf asdf asdfa dfasd fsad fsad fsa dfsdf.\\
\hline
\end{tabular} 
\end{table}

\begin{table}
\caption{Description of tutorial results.} \label{tbl:tutResults2}
  \centering
\begin{tabular}{|p{3cm}|p{3cm}|p{5cm}|}
\hline
\multicolumn{3}{|l|}{Compete system is now implemented.}\\
\hline
Tutorial 7&3&Hello assdafsadfsa dfsdfsdafdsf asdfsadfasdf sadfasdfa sdfsadfs adfasdfs dfasdfsdafsa dfsdf sadfasdf asdf asdfa dfasd fsad fsad fsa dfsdf.\\
\hline
Tutorial 8&3&Hello assdafsadfsa dfsdfsdafdsf asdfsadfasdf sadfasdfa sdfsadfs adfasdfs dfasdfsdafsa dfsdf sadfasdf asdf asdfa dfasd fsad fsad fsa dfsdf.\\
\hline
Tutorial 9&3&Hello assdafsadfsa dfsdfsdafdsf asdfsadfasdf sadfasdfa sdfsadfs adfasdfs dfasdfsdafsa dfsdf sadfasdf asdf asdfa dfasd fsad fsad fsa dfsdf.\\
\hline
Tutorial 10&3&Hello assdafsadfsa dfsdfsdafdsf asdfsadfasdf sadfasdfa sdfsadfs adfasdfs dfasdfsdafsa dfsdf sadfasdf asdf asdfa dfasd fsad fsad fsa dfsdf.\\
\hline
Tutorial 11&3&Hello assdafsadfsa dfsdfsdafdsf asdfsadfasdf sadfasdfa sdfsadfs adfasdfs dfasdfsdafsa dfsdf sadfasdf asdf asdfa dfasd fsad fsad fsa dfsdf.\\
\hline
\end{tabular}
\end{table}
\section{Deep Convolutional Neural Network test results}
\label{sec:DCNNresult}

This section describes the results obtained on testing a trained neural network on a test dataset. Tests is conducted on 3 neural networks trained on different datasets and compared with each other. This testing proses is conducted to find the neural network weights that will classified had written digits the most accurately. The neural networks is tested on a test dataset generated by grading 900 student tests, while extracting the character images. The answers from these tests is used to create labels for each digit image. Thus each 28 by 28 pixel digit image has a accompanied label specifies what that digit is. The database contains 16 000 labelled images. An additional dataset, called the MNIST dataset was also used in this proses. This dataset contains 60 000 images 
\section{Trained on generated database}
For a first attempt at training the neural network 10 000 of the generated database is used. The remaining 6000 digits is then used to test the accuracy of the network.

\subsection{Accuracy of network}

\subsection{Conclusion on accuracy}

\section{Trained on MNIST database}
For a first attempt at training the neural network 10 000 of the generated database is used. The remaining 6000 digits is then used to test the accuracy of the network.

\subsection{Accuracy of network}

\subsection{Conclusion on accuracy}


\section{Trained on mixed database}
For a first attempt at training the neural network 10 000 of the generated database is used. The remaining 6000 digits is then used to test the accuracy of the network.

\subsection{Accuracy of network}

\subsection{Conclusion on accuracy}











Attempt 2:

Accuracy for character recognition:

Using default deep neural network. V23

Test accuracy for network trained on MNIST dataset:
On test data for MNIST: 		0.9935
On test data for generated digits: 	0.666667

Main reason for error.
The MNIST dataset has images that are more consentrated to being binary.
Either black or white.

Test accuracy for network trained on generated digits:
On test data for MNIST:			 0.9462
On test data for generated digits:	 0.921569


Reasons The algorithm above does not work that well on the characters:

Accuracy for character recognition:

Using default deep neural network. Up to V22

Test accuracy for network trained on MNIST dataset:
On test data for MNIST: 		0.9935
On test data for generated digits: 	0.666667

Main reason for error.
The MNIST dataset has images that are more consentrated to being binary.
Either black or white.

Test accuracy for network trained on generated digits:
On test data for MNIST:			 0.9462
On test data for generated digits:	 0.921569


Reasons The algorithm above does not work that well on the characters:


Borders are not cut out properly.

Neural network focusses on color as well which decreases its accuracty for shape.
