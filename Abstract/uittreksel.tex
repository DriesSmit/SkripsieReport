%%% Thesis Introduction --------------------------------------------------

\begin{uittreksel}    %uncommenting this line, gives a different abstract heading
%\begin{opsomming}        %this creates the heading for the abstract page
Die doel van hierdie navorsings projek is om sagteware te ontwikkel wat automaties toetse, wat deur studente geskryf is, na te kan sien. Hierdie toetse word elkeen op 'n spesiale templaat geskryf. Die templaat laat die sagteware toe om die student se antwoord te vind nadat dit digitaal gekopieer is. Die standaard metode wat gebruik word in hierdie Optiesemerk-leser sagteware is om van inkleur rooster borrels gebruik te maak. Om 'n fout wat die student gemaak het regtemaak, moet die ou borrels eers uitgevee word and nuwe borrels ingekleur word. Hierdie prosses vat baie tyd en vermeeder die kanse dat 'n student die borrels verkeerd in kan vul. 
Die navorsingsprojek poog om hierdie probleem op te los deur addisionele sagteware te implementeer wat die gebruik van so 'n templaat vergemaklik. Met behulp van rekenaarvisie en twee masjienleertegnieke, naamlik Kunsmatige Neurale Netwerke en Probabilistiese Grafiese Modelle (PGM), kan 'n student nou vrae beantwoord deur karakters en borrels te gebruik. In die geval van 'n studentenommer hoef die student glad nie die borrels eers in te kleur nie. Die sagteware wat in hierdie projek geïmplementeer word, veroorsaak 'n verminderde tyd in antwoorde inskryf en verminder dus die aantal foute wat gemaak word. Nuwe metodes soos hierdie kan dus help dat meer OMR sagteware in die tradisionele onderwysstelsels ingebring kan word.
\end{uittreksel}