%%% Thesis Introduction --------------------------------------------------

\begin{uittreksel}    %uncommenting this line, gives a different abstract heading
%\begin{opsomming}        %this creates the heading for the abstract page
Die doel van hierdie navorsingsprojek is om sagteware te ontwikkel wat automaties toetse, wat deur studente geskryf is, na te sien. Die standaardmetode wat in hierdie Optiesemerk-leser sagteware gebruik word, is om van inkleur roosterborrels gebruik te maak. Om 'n fout wat die student gemaak het te korrigeer, moet die ou borrels eers uitgevee word. Hierdie proses is tydrowend en verhoog die kanse dat 'n student die borrels verkeerd invul. Hierdie navorsingsprojek poog om di\'e probleem op te los deur addisionele sagteware te implementeer, wat die gebruik van so 'n templaat vergemaklik. Met behulp van rekenaarvisie en twee masjienleertegnieke, naamlik Neurale Netwerke en Probabilistiese Grafiese Modelle, kan 'n student nou vrae beantwoord deur borrels uitvee, asook antwoorde in karakters te skryf.  Die sagteware wat in die projek ontwikkel word, maak die tipe toetse makliker om te gebruik, terwyl nogsteeds 'n ho\"e akuraatheid in toetsresultate verkry word.
\end{uittreksel}