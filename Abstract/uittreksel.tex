%%% Thesis Introduction --------------------------------------------------

\begin{uittreksel}    %uncommenting this line, gives a different abstract heading
%\begin{opsomming}        %this creates the heading for the abstract page
Die doel van hierdie navorsingsprojek is om sagteware te ontwikkel wat automaties toetse, wat deur studente geskryf is, na te sien. Hierdie toetse word elkeen op 'n spesiale templaat geskryf. Die templaat stel die sagteware in staat om die student se antwoord te vind nadat dit digitaal gekopieer is. Die standaardmetode wat in hierdie Optiesemerk-leser sagteware gebruik word. is om van inkleur roosterborrels gebruik te maak. Om 'n fout wat die student gemaak het te korrigeer, moet die ou borrels eers uitgevee word and nuwe borrels ingekleur word. Hierdie prosses is tydrowend en verhoog die kanse dat 'n student die borrels verkeerd invul. Dié navorsingsprojek poog om hierdie probleem op te los deur addisionele sagteware te implementer, wat die gebruik van so 'n templaat vergemaklik. Met behulp van rekenaarvisie en twee masjienleertegnieke, naamlik Neurale Netwerke en Probabilistiese Grafiese Modelle (PGM), kan 'n student nou vrae beantwoord deur karakters tesame met borrels te gebruik. In die geval van die studentenommer, hoef die student nie die borrels in te kleur nie. Die sagteware wat in hierdie projek geïmplementeer word, verkort die tyd vir die inskryf van antwoorde en sodoende verminder die aantal foute wat moontlik gemaak word. Nuwe metodes soos hierdie kan dus help dat meer OMR sagteware in die tradisionele onderwysstelsels ingebring kan word.
\end{uittreksel}