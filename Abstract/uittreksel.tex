%%% Thesis Introduction --------------------------------------------------

\begin{uittreksel}    %uncommenting this line, gives a different abstract heading
%\begin{opsomming}        %this creates the heading for the abstract page
Die doel van hierdie navorsings projek is om sagteware te ontwikkel wat automaties toetse, wat deur studente geskryf is, na te kan sien. Hierdie toetse word elkeen op 'n spesiale templaat geskryf. Hierdie templaat laat die sagteware toe om die student se antwoord te vind nadat dit digitaal gekopieer is. Die standaard metode wat gebruik word in hierdie Optiesemerk-leser sagteware is om inklear rooster borrels to gebruik. Om 'n fout wat die student gemaak het regtemaak, moet die ou borrels eers uitgevee word and nuwe borrels ingekleur word. Hierdie prosses vat baie tyd en vermeeder die kanse dat 'n student die borrels verkeerd in kan vul. Hierdie navorsings projek probeer die probleem aan pak deur om additionele metodes in te bou, wat die gebruik van die templaat vergemaklik. Een van die metodes is om die student die vryheid te gee om 'n antwoord dood te karp i.p.v. dit uittevee. Dit spaar die student die tyd wat sou bestee word op die antwoord uitvee. Deur om van twee masjien leer tegnieke gebruik te maak, naamlik Kunsmatige Neurale Netwerke(KNN) en Probabilistiese Grafiese Modelle(PGM), word die student die vryheid gegee om sy/haar studente nommer net met karakters uit te skryf, sonder borrels. Verder laat die stelsel ook toe vir 'n student om sy/haar ander antwoorde in borrels en karakters uit te skryf. Dit karakters word dan vergelyk met die borrels wat ingekleur is om op 'n finale antwoord te besluit. Al hierdie additionele metodes laat die student toe om vinniger die toets se antwoorde intevul en dus minder kans het om 'n fout te maak in die proses. Dus metodes, soos beskryf in hierdie verslag, laat toe dat onderwys institusies makliker Optiesemerk-leser sagteware implementeer en gebruik. 
\end{uittreksel}