
% Thesis Abstract -----------------------------------------------------


%\begin{abstractslong}    %uncommenting this line, gives a different abstract heading
\begin{abstracts}        %this creates the heading for the abstract page
\nomenclature[A]{$OMR$}{Optical mark recognition}
\nomenclature[pgmAcronym]{$PGM$}{Probabilistic graphical model}
\nomenclature[A]{$ANN$}{Artificial neural network}
The aim of this research project is to develop software that can automatically grade tests, written by students. These tests are written on a special template. This template allows the software to extract the students intended answers, after it has been scanned into a digital form. The standard method used in these Optical Mark Recognition(OMR) software, is to only allow the use a grid of bubbles to colour in answers. To correct a mistake made by a student, the previous answer bubbles must first be erased and new ones coloured in. This takes time and increases the probability that a learner might colour in bubbles incorrectly. This research project tries to solve this problem by including additional features that eases the use of such a template. A student is allowed to cross out an answer instead of erasing it. This reduces the need for erasing answers. Using two machine learning techniques, namely Artificial Neural Networks(ANN) and Probabilistic Graphical Models(PGM), a student is allowed to fill in his/her student number using only characters. Further features that the system also incorporates are to allow students to write their answers in characters. These characters are then cross-referenced with the bubble grid. These additional features allows for a decreased writing time and an increase in accuracy in filling in these templates. New methods like these thus allows OMR templates to more easily be incorporated into the traditional educational systems.
\end{abstracts}
%\end{abstractlongs}
