
% Thesis Abstract -----------------------------------------------------


%\begin{abstractslong}    %uncommenting this line, gives a different abstract heading
\begin{abstracts}        %this creates the heading for the abstract page
\nomenclature[A]{$OMR$}{Optical mark recognition}
\nomenclature[pgmAcronym]{$PGM$}{Probabilistic graphical model}
\nomenclature[A]{$NN$}{Neural network}
The aim of this research project is to develop software that can automatically grade tests, written by students on a special template. This template allows the software to extract the students intended answers, after it has been scanned into a digital form. The standard method used in these Optical Mark Recognition (OMR) software, is to allow a learner to specify answers by colouring in bubble grids. To correct a mistake, the previous answer bubbles must first be erased and new ones coloured in. This takes time and increases the probability that a learner might colour in bubbles incorrectly. 

This research project tries to solve this problem by implementing additional software that eases the use of such a template. Using computer vision and two machine learning techniques, namely Neural Networks (NN) and Probabilistic Graphical Models (PGM), a student can now answer questions using characters and bubbles. In the case of a student number the student does not need to colour in the bubbles at all. The software implemented in this project allows for a decreased time in filling in these templates and thus decreases the number of mistakes made. The methods proposed here thus simplifies the incorporation of OMR templates into the traditional education system.
\end{abstracts}
%\end{abstractlongs}
