\chapter{Outcome compliance}
\label{ap:outCompliance}
\graphicspath{{Appendix2/Appendix2figures/}}
Tables \ref{tbl:ECSATable1} and \ref{tbl:ECSATable2} describes the required ECSA Exit Level Outcomes and how this project adheres to these outcomes.
\begin{table}
\caption{Disrcption of Exit level outcomes and how this project adherse to them.} \label{tbl:ECSATable1}
\begin{tabular}{|p{6cm}|p{3cm}|p{6cm}|}
\hline
\textbf{Outcome}&\textbf{Reference}&\textbf{Description}\\
\hline
1. Problem solving: Identify, formulate, analyse and solve complex engineering problems creatively and innovatively. & \textit{1,2,3,4} & A detailed formulation was needed to be made in Chapter 1. Further improvements using two machine learning techniques was also introduced and implemented.\\
\hline
2. Application of scientific and engineering knowledge: Apply knowledge of mathematics, natural sciences,
engineering fundamentals and an engineering speciality to solve complex engineering problems. & \textit{1, 3 \& 4} & Engineering knowledge obtained in the degree allowed for the understanding and implementation of mathematical concepts such as Neural Networks, Radon transforms and Probabilistic Graphical Models. Further computer programming knowledge was needed to implement and optimize the software for increased speed.\\
\hline
3. Engineering Design: Perform creative, procedural and non-procedural design and synthesis of components, systems,
engineering works, products or processes. & \textit{1, 3 \& 4} & In the design process of implementing Image Processing and Neural Networks a procedural process is demonstrated. Implementing a Probabilistic Graphical Model demonstrates a creative approach to solve this problem\\
\hline
5. Engineering methods, skills and tools, including Information Technology: Demonstrate competence to use
appropriate engineering methods, skills and tools, including those based on information technology. &\textit{1.4, 1.5 \& 5}  & Engineering methods used in this project include agile development, software version control, through Git, and project scheduling to efficiently manage time.        \\
\hline
\end{tabular}
\end{table}
\begin{table}
\caption{Disrcption of Exit level outcomes and how this project adherse to them.} \label{tbl:ECSATable2}
\begin{tabular}{|p{6cm}|p{3cm}|p{6cm}|}
\hline
\textbf{Outcome}&\textbf{Reference}&\textbf{Description}\\
\hline
6. Professional and technical communication: Demonstrate competence to communicate effectively, both orally and
in writing, with engineering audiences and the community at large. & \textit{All}  &   This outcome is demonstrated in the report and oral presentation.       \\
\hline
9. Independent Learning Ability: Demonstrate competence to engage in independent learning through well-developed
learning skills.&\textit{2,3 \& 4}  &  Independent learning was continuously required in understanding new concepts in each stage of the project's design process. These concepts includes Neural Networks, Radon transforms, Probabilistic Graphical Models and Image Processing.\\
\hline
\end{tabular}
\end{table}
% ------------------------------------------------------------------------

%%% Local Variables:
%%% mode: latex
%%% TeX-master: "../thesis"
%%% End:
